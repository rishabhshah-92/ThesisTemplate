%Everything before the \begin{document} command is known as a "Preamble". It does not show up in the output document, but this is essentially the control center of your document. This is where you define your default font style and font size (everything can be changed later, as needed), define your bibliography style, import different packages like a package for writing chemical equations, a package for including click-able internet links, etc.

\documentclass[12pt]{report} %set default font size and document type. This should be the first executable line of the preamble.

%%%%%% TYPESETTING PACKAGES %%%%%%%
\usepackage[margin=1in]{geometry} %set margins of 1 inch
\usepackage[english]{babel} %for specifying language rules (I think; unsure why this is required)
\usepackage[hidelinks,bookmarksopen=true,bookmarksnumbered=true]{hyperref} % to make all citations active links and hide hideous green borders around them
\usepackage[font=small, figurename=Fig.]{caption} %for figure and table caption font style. If not used, default font will be used.
\renewcommand{\thefootnote}{\fnsymbol{footnote}}%allows footnotes with special symbols
\usepackage{graphicx} %to import figures
\usepackage{setspace} %to dynamically change line spacing, if needed.
\doublespacing %default line spacing of entire document set to double-spaced (typical for dissertations)
\usepackage{parskip} %new paragraph is separated from previous paragraph by additional vertical spacing, in addition to typical horizontal indentation of first line
\setlength{\parindent}{15pt}
\usepackage{etaremune} %reversed enumeration

%%%%%% MATH PACKAGES %%%%%%%
\usepackage{amsmath,esint} %for math

%%%%%% BIBLIOGRAPHY PACKAGES %%%%%%%
\usepackage[compress,authoryear]{natbib} %this is the basic bibliography package, and the [arguments] specify citation style i.e., author-year versus superscripted number, etc.
\usepackage[sectionbib]{chapterbib} %this is to make sure each chapter's references show up at the end of the chapter, and not all together at the end of the thesis
\usepackage{bibentry} %package for various different ways of placing in-text citations
\bibliographystyle{apalike} %bibliography style used in the "references" section

%%%%%% PDF OUTPUT PACKAGES %%%%%%%
\usepackage{bookmark} %for having clickable bookmarks in the output PDF outline

%%%%%% CHEMISTRY PACKAGES %%%%%%%
\usepackage{chemmacros} %to write chemical reactions
\usepackage{chemfig} %to draw chemical structures

%%%%%% OTHER, OPTIONAL PACKAGES %%%%%%%
%NOTE: these are just some example packages you might need. There are lots more packages available for variety of needs

%\usepackage{lipsum} % for creating fake text as a placeholder
%\usepackage{tikz} % for creating graphics such as flowcharts
%\usepackage{tcolorbox} %for highlighting text
%\usepackage{xcolor} %for using color names
%\usepackage{siunitx} %for writing numbers and units consistent with the SI system
\usepackage{gensymb} %symbols like degree for temperature
%\usepackage{wrapfig} %to wrap text around a small figure
%\usepackage{framed} %frame figures with a box

\begin{document}

% Setting page number to Roman because Arabic numbering usually does not begin until after all the preface items e.g., abstract, table of contents, etc.
\pagenumbering{roman}

%Creating a bookmark so the output PDF will have this section show up in the outline panel.
\pdfbookmark[section]{Title Page}{Title Page}

%The include command calls another .tex file.
\thispagestyle{empty}
\begin{center}
\begin{singlespacing}
\vspace*{3\baselineskip}
\textbf{A \LaTeX{}  Template for a Master's or PhD Thesis:\\ \noindent
Because Microsoft Word is terrible for scientific writing\\}
\vspace{4.5\baselineskip}
Submitted in partial fulfillment of the requirements for\\\vspace{0.75\baselineskip}the degree of\\\vspace{0.75\baselineskip}Some Degree\\\vspace{0.75\baselineskip}in\\\vspace{0.75\baselineskip}Blahblah Engineering\\
\vspace{5\baselineskip}
Student Name\\
\vspace{2\baselineskip}
B.S., Basic Engineering, Whatever University.\\
M.S., Next-level Engineering, Random University\\
\vspace{8.5\baselineskip}
Name of Curent University\\
City, State\\
\vspace{\baselineskip}
Month Year
\end{singlespacing}
\end{center}

%The following generates a single page with a copyright statement. Feel free to erase/comment out.
\newpage
\thispagestyle{empty}
\vspace*{20\baselineskip}
\begin{center}
{\small \textcopyright} Student Name, Year\\All rights reserved
\end{center}
\newpage

\pdfbookmark[subsection]{Acknowledgements}{Acknowledgements}
%The first page here is just a "dedication" page. You can remove it by erasing/commenting everything up until the start of the Acknowledgements chapter.
\thispagestyle{empty}
\vspace*{\fill}
\hspace*{\fill}\textit{Dedicated to dogs}\hspace*{\fill}
\vspace*{\fill}
\newpage
\chapter*{Acknowledgements}
I would like to thank my advisor, my friends, my family, my funding sources, etc.

\pdfbookmark[section]{Abstract}{Abstract}
\chapter*{Abstract}
%The abstract is typically not included as a numbered section. Hence, putting an asterisk after the \chapter command prevents this section from being numbered.

\LaTeX{} is an extremely powerful and efficient tool for scientific writing. However, as with all software, there is a bit of a learning curve involved. With \LaTeX{}, the biggest (and perhaps only) learning curve is that it does not have graphical user interface. All formatting commands such as \textbf{boldfacing}, \textit{italicizing}, {\tiny changing font size on the fly}, text\textsubscript{subscripting}, $m=a^t_h$, etc. have to be entered as plain-text code: \verb|\textbf{boldfacing}|, \verb|\textit{italicizing}|, \verb|{\tiny changing font size on the fly}|, \verb|text\textsubscript{subscripting}|, \verb|$m=a^t_h$|. Seems cumbersome, right? Why type the extra commands, when you can instead select the text you wish to modify, and just click a button in your writing software?

The thing is, scientific documents such as manuscripts, dissertations, and books are often ``complex''. What I mean by that is they are not just text documents. They have plenty of figures, tables, equations, chapters, sections, subsections, literature references etc. In theory, typical writing software like MS Word and Libre Writer have the capability to handle this. But in practice, as the complexity of a document increases, these software are prone to several issues:

\begin{enumerate}
\item Crashing without auto-saving, because of too much memory/processor load
\item Messing up numbering of figures, tables, sections, equations, etc. when document is opened.
\item Compatibility issues with citation software like Mendeley.
\item Complex equations written in companion software like MathType (for MS Word) and Math Formula (for Libre Writer) just don't look aesthetic. The fonts look different, the subscripting and superscripting is all off, and ... well it just doesn't look pretty.
\item Inter-computer compatibility issues i.e., the document looks a certain way on your computer, but when you send it to someone else, it looks different on their computer for a variety of reasons such as you have a Mac, but they have Windows.
\item Updating in-text references to figures, tables, etc. is cumbersome and requires user to manually double-check every one of them even though Word is supposed to do it automatically.
\end{enumerate}

Instead, \LaTeX{} simplifies things incredibly. Here's a point-by-point list of how \LaTeX{} is a solution to these issues:

\begin{enumerate}
\item The reason Word crashes is that it is not only trying to keep the text of the document on memory (while you are editing the file), but it is also trying to keep all formatting, figures, references, equations, captions, etc. preserved while letting you edit them in real time. Instead, \LaTeX{} separates these two processes. The user only enters the text and the commands for formatting, importing figures, etc. are entered as plain text. All of this knitted together only when the \LaTeX{} code is compiled, and the output is a clean PDF. The few seconds it takes for \LaTeX{} to compile a code is the only time during which your machine's processing power is used by \LaTeX{}.
\item The reason Word messes up numbering is that it refreshes all numbering when the file is first opened. Instead, \LaTeX{} generates a PDF wherein all numbering is ``locked in'', and not refreshed when the PDF file is opened.
\item \LaTeX{} has no need to ``talk'' to your citation manager.
\item \LaTeX{} uses the same font for its text, as it does for its math. And \LaTeX{} is built to accept math commands of a wide range of complexities.
\item \LaTeX{} generates a PDF, which by definition is a ``portable document format'' and thus it looks exactly the same on all computers, irrespective of what OS they are using (Mac, Linux, Windows, etc.), what PDF reader they are using, etc.
\item \LaTeX{} only processes cross-references at the time of compiling, and thus relieves the user of the task of double-checking the referencing (as long as there were no typos in the referencing commands in the first place).
\end{enumerate}


\pdfbookmark[section]{Contents}{Contents}
\tableofcontents

\pdfbookmark[section]{List of Figures}{List of Figures}
\listoffigures

\begingroup
\pdfbookmark[section]{List of Tables}{List of Tables}
\listoftables
\endgroup

\newpage
\pagenumbering{arabic}

\chapter{Chapter name}
\section{A section on math}\label{math-sec}
To activate in-line math mode, enclose math within dollar signs. So, if your code has \verb|$\alpha^2 = 10\beta_{s}^{0.8}$|, \LaTeX{} will interpret it as $\alpha^2 = 10\beta_{s}^{0.8}$.

You can also insert complicated math using the \verb|align| environment like below:

\begin{align*}
  \iiint\limits_V(\nabla \cdot \mathbf{F}) dV
      & = \oiint \limits_{S(V)} \mathbf{F \cdot \hat{n}} dS \\
  \iiint\limits_V(\nabla \times \mathbf{F}) dV
      & = \oiint \limits_{S(V)} \mathbf{\hat{n} \times F} dS \\
  \iiint\limits_V(\nabla f) dV
      & = \oiint\limits_{S(V)}\mathbf{\hat{n}}f dS
\end{align*}

If you want the equations to be numbered, use the \verb|\begin{align}| command instead of \verb|\begin{align*}|. The \verb|\label| command declares the unique key to be used for referring to this equation, as shown in Equations \ref{eq-omega1} and \ref{eq-omega2}.

\begin{align}
S(\omega)
&= \frac{\alpha g^2}{\omega^5} e^{[ -0.74\bigl\{\frac{\omega U_\omega 19.5}{g}\bigr\}^{\!-4}\,]}\label{eq-omega1} \\
&= \frac{\alpha g^2}{\omega^5} \exp\Bigl[ -0.74\Bigl\{\frac{\omega U_\omega 19.5}{g}\Bigr\}^{\!-4}\,\Bigr]\label{eq-omega2} 
\end{align}

\section{A section on citing literature}\label{citing-sec}
\subsection{Preparing your .bib file} \label{sec:prepping_bib}
To include citations, first add them to the \verb|ch01_REFS.bib| file in a specific format called ``BibTeX'' format. The easiest way to do this is from a citation managing software (e.g., in Mendeley, right-click on any paper and select ``Copy As $ > $ Bibtex entry''. Next, open the \verb|ch01_REFS.bib| file in your TeX editor, and paste the BibTeX entry at the end. It should look something like this:

\begin{center}
\includegraphics[width=\linewidth]{"ch01/bib_example"}
\end{center}

The \verb|Shah2020| that appears at the top of the BibTeX record is called a ``cite key''.
\textbf{Warning:} No two records in the \verb|ch01_REFS.bib| can have the same cite key!

\subsection{Get citing}
To cite the study shown in Section \ref{sec:prepping_bib}, you should type the command \verb|\citep{Shah2020}|. Sometimes you may want to cite a paper at the end of a sentence \citep{Shah2020}. But sometimes, you might also insert a citation as part of sentence, such as \citet{Shah2020}. Sometimes, you might want to cite multiple references altogether inside one set of parantheses \citep{Shah2018,Shah2020,Robinson2019}. \LaTeX{} identifies papers by the same author, and compresses them together i.e., instead of (Shah et al. 2018, Shah et al. 2020), it writes \citep{Shah2018,Shah2020}. Sometimes, you may want to cite a study in text that is already in parantheses (e.g., the study by \citep{Shah2018} was about air pollution). If you want, you can avoid double parantheses as follows: \cite[e.g., the study by][was about air pollution]{Shah2018}.

\section{A section on embedding figures}\label{figures-sec}
As shown in Figures \ref{fig:plot-label} and \ref{fig:plot2-label}, data visualization is an art. The code for importing figures is fairly self-explanatory.

\begin{figure}[!h]
\centering
\includegraphics[width=0.7\linewidth]{"ch01/Some_Plot"}
\caption[A figure]{This is a detailed figure caption, describing what the figure shows. A general good practice: a figure and its caption, together, should be self-sufficient in conveying the message. The reader should not have to look for further explanation of the figure in the text.}
\label{fig:plot-label}
\end{figure}

The \verb|[!h]| flag after the \verb|\begin{figure}| command forces \LaTeX{} to place the Figure exactly where it appears in the code. If this flag is not included, \LaTeX{} will try to place the Figure where it ``fits'' best. Other options instead of \verb|!h| include \verb|!b| (bottom of page), \verb|!t| (top of page), etc.

The \verb|\caption| command takes two arguments: the first one (in square brackets) is a short caption, while the second one (in curly brackets) is a detailed caption. The short caption appears in the List of Figures. The detailed caption appears directly beneath the Figure.

As always, the \verb|\label| command declares the unique key to be used for referring to this Figure.

\begin{figure}[!h]
\centering
\includegraphics[width=0.7\linewidth]{"ch01/Some_Other_Plot"}
\caption[Another figure]{This is a detailed figure caption for another figure, describing what the figure shows. Fun fact: this figure was prepared in RStudio open-source software, just like Figure \ref{fig:plot-label} (yes, you can have a reference to another Figure inside a Figure caption; try doing that in Word!).}
\label{fig:plot2-label}
\end{figure}

\section{A section on tables}\label{tables-sec}
A table can be created pretty easily in \LaTeX{}, as shown in Table \ref{tab:simpletable}. The number of times $\vert$\verb| c |$\vert$ is entered after the \verb|\begin\{tabular}| command declares the number of columns to be used. The \verb|c| stands for ``center'' justification. Other options can be \verb|l| for left, and \verb|r| for right.

As with Figure captioning, there is a short caption (in square brackets; this shows up in List of Tables), and a long caption (in curly brackets; this shows up at the top of the Table). While Figure captions are typically placed below the figure, the convention for Table captions is to place them on top of Tables. This can be achieved by simply moving the \verb|\caption| command as shown in the code for Table \ref{tab:complextable}.

\begin{table}[!h]
\begin{center}
  \begin{tabular}{| c | c | c |}
    \hline
    1 & 2 & 3 \\ \hline
    4 & 5 & 6 \\ \hline
    7 & 8 & 9 \\
    \hline
  \end{tabular}
\caption[A small, simple table]{A table of alphabets. Though this caption is supposed to be detailed, I cannot think of a way to put more text here.}
\label{tab:simpletable}
\end{center}
\end{table}

While Table \ref{tab:simpletable} was fairly simple, a more complicated table with varying column widths can also be created, as shown in Table \ref{tab:complextable}.

\begin{table}[!h]
\caption[A slightly more complex table]{A table of numbers and things. Though this caption is supposed to be detailed, I cannot think of a way to put more text here.}
\label{tab:complextable}
\begin{tabular}{|p{20mm}|p{40mm}|p{30mm}|p{10mm}p{10mm}p{10mm}p{10mm}|}
\hline
\textbf{City}&\textbf{Neighborhood}&\textbf{Time period}&\multicolumn{4}{c|}{\textbf{Results}}\\
&	&	& $\alpha$&$\beta$&$\chi^2$&$\mu$ [m]\\
\hline
Oakland&Full domain&All day&1.83&-0.21&0.89&218\\
 &Downtown&All day&1.24&-0.04&0.03&36\\
 &Urban residential&All day&2.09&-0.23&0.67&234\\
 \hline
Pittsburgh&Full domain&All day&2.18&-0.27&0.52&268\\
 & &Morning&6.42&-0.39&0.42&398\\
 &	 &Midday&0.55&-0.14&0.05&148\\
 & &Afternoon&4.08&-0.42&0.54&419\\
\hline
\end{tabular}
\end{table}

\section{A section on footnotes}\label{sec-footnotes}
I say, ``Footnotes are so easy!''\footnote{And you say, ``How easy are they?!''}

\section{A section on bulleting and numbering}\label{sec-itemizing}
Bulleted lists are very simple to prepare in \LaTeX{}, using the \verb|itemize| environment.
\begin{itemize}
\item An item
\item Another item
\item And so forth
\end{itemize}

The only difference between numbered and bulleted lists is the \verb|enumerate| environment instead of \verb|\itemize|. \LaTeX{} takes care of the numbering during code compilation.
\begin{enumerate}
\item An item
\item Another item
\item And so forth
\end{enumerate}

If you need the numbering reversed, use the \verb|\etaremune| environment\footnote{see the brilliance of it? ``etaremune'' is ``enumerate'' spelled backwards.}.
\begin{etaremune}
\item An item
\item Another item
\item And so forth
\end{etaremune}

\section{A section on chemical reactions}\label{chem-reac-sec}
You can write chemical reactions using the \verb|\chemfig| package:
\begin{center}
\schemestart
\chemfig{H_2O} \arrow{->[$\lambda=185$ nm]}[,1.5] \chemfig{OH} + \chemfig{H}
\schemestop
\end{center}
\begin{center}
\schemestart
\chemfig{O_2} \arrow{->[$\lambda=185$ nm]}[,1.5] 2 O(\textsuperscript{3}P) \arrow{->[\chemfig{O_2}]} 2 O\textsubscript{3} \arrow{->[$\lambda=254$ nm]}[,1.5] 2 O(\textsuperscript{1}D) \arrow{->[\chemfig{H_2O}]} \textit{n}OH
\schemestop
\end{center}

You can also write chemical reactions using the \verb|\chemmacros| package, which follows a slightly different syntax. Unfortunately, neither of them will check your reactions for stoichiometric balancing, so that part's on you.
\begin{center}
\ch{Na2SO4 ->[ H2O ] 2 Na+ + SO4^2-}
%be sure to leave an empty line otherwise the two reactions will be mixed together

\ch{( 2 Na+ ,SO4^2- ) + (Ba^2+ , 2 Cl- ) -> BaSO4 v + 2 NaCl}
\end{center}

You can also include chemical reactions (and organic structures) as a Figure, as shown in Figure \ref{fig:FuncFrag}. \textit{Note:} drawing organic structures can be time-consuming if you have too many of these. If you are writing a hard-core chemical engineering/chemistry document, you might be better off drawing the reactions in other specific software and then import them as figures into your \LaTeX{} document.

\begin{figure}[h]
\centering
\schemestart
\chemfig{OH}
+ \chemfig{R\textsubscript{1}-[:25,0.5]-[:-25,0.5]-[:25,0.5](-[:90,0.5])-[:-25,0.5]-[:25,0.5]-[:-25,0.5]R\textsubscript{2}}
\arrow(reac.mid east--func.mid west){->[functionalization]}[30,2]
\chemname{\chemfig{R\textsubscript{1}-[:25,0.5]-[:-25,0.5](=[:90,0.5]O)-[:25,0.5](-[:90,0.5])-[:-25,0.5](-[:-90,0.5]OH)-[:25,0.5]-[:-25,0.5]R\textsubscript{2}}}{lower volatility}
\arrow(@reac.mid east--frag.mid west){->[fragmentation]}[-30,2]
\chemname{\chemfig{R\textsubscript{1}-[:25,0.5]-[:-25,0.5](=[:90,0.5]O)}
+
\chemfig{-[:25,0.5](=[:90,0.5]O)-[:-25,0.5]-[:25,0.5]-[:-25,0.5]R\textsubscript{2}}}{higher volatility}
\schemestop
\caption[Short caption for chemistry]{Caption describing what is going on in this chemical reaction.}
\label{fig:FuncFrag}
\end{figure}


\newpage
\begin{footnotesize}\singlespacing
\renewcommand{\bibname}{References}
\bibliographystyle{apalike}
\addcontentsline{toc}{section}{References}
\bibliography{ch01/ch01_REFS}
\end{footnotesize}


\end{document}
\documentclass[12pt]{report} %set default font size and document type. This should be the first executable line of the preamble.
\usepackage{url}

\begin{document}
\section*{User Manual for this \LaTeX{} template}
\textit{Rishabh Shah}
\subsection*{Getting started}
Thank you for downloading my \LaTeX{} thesis template. Before you can use this template, you must make sure of the following:
\begin{enumerate}
\item You must have a PDF reader installed on your machine. On a Mac, ``Preview'' is the default PDF reader. On Windows, if you don't have a PDF reader, you can download Foxit PDF reader for free here: \url{https://www.foxitsoftware.com/pdf-reader/}
\item You must have LaTeX installed on your machine. The best way to do this is by installing the MiKTeX bundle from \url{https://miktex.org/download}. This will install the base LaTeX engine, as well as some TeX editors on your machine.
\end{enumerate}

\subsection*{A few tests to make sure things are in place}
\begin{enumerate}
\item \textit{Let's test if a PDF reader installed correctly on your machine:} in the ThesisTemplate folder you downloaded, open the "TemplateOutput.pdf" file. It should open in your default PDF reader.
\item \textit{Let's test whether LaTeX installed correctly or not:}
\begin{enumerate}
\item Make sure the ThesisTemplate folder is entirely downloaded on your local machine. It does not matter if you place it on your desktop, or your downloads folder, or wherever.
\item If you have a Mac, doubleclick the ``MAC\_{}MakeThesis.sh'' file (``WIN\_{}MakeThesis.bat'' for Windows). A Terminal (Mac) or Command Line (Windows) window should open up at this point and you will start seeing several lines of code auto-scrolling.
\item If you have never installed/run \LaTeX{} before today, the first execution of the MakeThesis command might take a while depending on your Internet connection. This is the only time this template will require an Internet connection, because it needs to download several \LaTeX{} packages such as math, chemistry, etc. If everything works fine, subsequent executions of the MakeThesis command will not require an Internet connection.
\item If the MakeThesis command is able to run successfully, several things should happen:
- the Terminal/Command Line window should auto-close
- a new PDF file called "MyThesis.pdf" should be generated in the ThesisTemplate folder, and should also auto-open.
\end{enumerate}

\end{enumerate}
If all of this works, you're all set with regards to installation. You are now ready to start editing this template to make your own thesis!

\subsection*{The structure of the thesis}
The folder \verb|AllOtherCrap| contains all the content files that are used by \LaTeX{} in building your thesis. The \verb|Thesis\_{}Central.tex| file is the master file in which you control various aspects of your thesis such as chapter ordering, font style, formatting, paragraph styling, importing new packages, etc. During code compilation, this master file will gather all the other chapters such as the \verb|Abstract.tex|, \verb|TitleAndCopyright.tex|, etc.

Each chapter is placed in a separate subfolder for sake of cleanliness. Each chapter's subfolder also contains the supporting files used in that chapter i.e., figures and references.

\subsection*{Editing chapters}
Editing chapters is fairly straightforward: open the chapter's .TeX file (e.g., \verb|ch01.tex|) in whatever \LaTeX{} editor you use, and start editing it! When you're done, make sure to hit save before closing the .TeX file. 

If you wish to add new references to a chapter, you must do so separately by editing that chapter's corresponding .bib file (e.g., \verb|ch01.bib|), as instructed in the ``TemplateOutput.pdf'' file.

If you wish to add new figures, you must place the figure in the relevant chapter's subfolder, and refer to them in that chapter's .TeX file using the \verb|\includegraphics| command, as demonstrated in the .TeX files provided with this template (e.g., \verb|ch01.tex|).

To see how your edits look in the final typeset version of the thesis, go to the ``ThesisTemplate'' folder and execute the MAC\_{}MakeThesis.sh file (if you are using Windows, execute the WIN\_{}MakeThesis.bat) by double-clicking on it.

\subsection*{Adding new chapters}
To add a new chapter, create a new subfolder and name it whatever you like (for sake of consistency, might as well call it \verb|ch03|). To be successfully included in the final thesis, each chapter's subfolder must have a .TeX file (e.g., \verb|ch03.tex|), a .bib file (e.g., \verb|ch03_REFS.bib|), and whatever figures you wish to embed in that chapter.

Next, in the \verb|AllOtherCrap| folder, open the \verb|Thesis\_{}Central.tex| file. Scroll down and find the chapter inclusion commands (\verb|\chapter{Chapter name}
\section{A section on math}\label{math-sec}
To activate in-line math mode, enclose math within dollar signs. So, if your code has \verb|$\alpha^2 = 10\beta_{s}^{0.8}$|, \LaTeX{} will interpret it as $\alpha^2 = 10\beta_{s}^{0.8}$.

You can also insert complicated math using the \verb|align| environment like below:

\begin{align*}
  \iiint\limits_V(\nabla \cdot \mathbf{F}) dV
      & = \oiint \limits_{S(V)} \mathbf{F \cdot \hat{n}} dS \\
  \iiint\limits_V(\nabla \times \mathbf{F}) dV
      & = \oiint \limits_{S(V)} \mathbf{\hat{n} \times F} dS \\
  \iiint\limits_V(\nabla f) dV
      & = \oiint\limits_{S(V)}\mathbf{\hat{n}}f dS
\end{align*}

If you want the equations to be numbered, use the \verb|\begin{align}| command instead of \verb|\begin{align*}|. The \verb|\label| command declares the unique key to be used for referring to this equation, as shown in Equations \ref{eq-omega1} and \ref{eq-omega2}.

\begin{align}
S(\omega)
&= \frac{\alpha g^2}{\omega^5} e^{[ -0.74\bigl\{\frac{\omega U_\omega 19.5}{g}\bigr\}^{\!-4}\,]}\label{eq-omega1} \\
&= \frac{\alpha g^2}{\omega^5} \exp\Bigl[ -0.74\Bigl\{\frac{\omega U_\omega 19.5}{g}\Bigr\}^{\!-4}\,\Bigr]\label{eq-omega2} 
\end{align}

\section{A section on citing literature}\label{citing-sec}
\subsection{Preparing your .bib file} \label{sec:prepping_bib}
To include citations, first add them to the \verb|ch01_REFS.bib| file in a specific format called ``BibTeX'' format. The easiest way to do this is from a citation managing software (e.g., in Mendeley, right-click on any paper and select ``Copy As $ > $ Bibtex entry''. Next, open the \verb|ch01_REFS.bib| file in your TeX editor, and paste the BibTeX entry at the end. It should look something like this:

\begin{center}
\includegraphics[width=\linewidth]{"ch01/bib_example"}
\end{center}

The \verb|Shah2020| that appears at the top of the BibTeX record is called a ``cite key''.
\textbf{Warning:} No two records in the \verb|ch01_REFS.bib| can have the same cite key!

\subsection{Get citing}
To cite the study shown in Section \ref{sec:prepping_bib}, you should type the command \verb|\citep{Shah2020}|. Sometimes you may want to cite a paper at the end of a sentence \citep{Shah2020}. But sometimes, you might also insert a citation as part of sentence, such as \citet{Shah2020}. Sometimes, you might want to cite multiple references altogether inside one set of parantheses \citep{Shah2018,Shah2020,Robinson2019}. \LaTeX{} identifies papers by the same author, and compresses them together i.e., instead of (Shah et al. 2018, Shah et al. 2020), it writes \citep{Shah2018,Shah2020}. Sometimes, you may want to cite a study in text that is already in parantheses (e.g., the study by \citep{Shah2018} was about air pollution). If you want, you can avoid double parantheses as follows: \cite[e.g., the study by][was about air pollution]{Shah2018}.

\section{A section on embedding figures}\label{figures-sec}
As shown in Figures \ref{fig:plot-label} and \ref{fig:plot2-label}, data visualization is an art. The code for importing figures is fairly self-explanatory.

\begin{figure}[!h]
\centering
\includegraphics[width=0.7\linewidth]{"ch01/Some_Plot"}
\caption[A figure]{This is a detailed figure caption, describing what the figure shows. A general good practice: a figure and its caption, together, should be self-sufficient in conveying the message. The reader should not have to look for further explanation of the figure in the text.}
\label{fig:plot-label}
\end{figure}

The \verb|[!h]| flag after the \verb|\begin{figure}| command forces \LaTeX{} to place the Figure exactly where it appears in the code. If this flag is not included, \LaTeX{} will try to place the Figure where it ``fits'' best. Other options instead of \verb|!h| include \verb|!b| (bottom of page), \verb|!t| (top of page), etc.

The \verb|\caption| command takes two arguments: the first one (in square brackets) is a short caption, while the second one (in curly brackets) is a detailed caption. The short caption appears in the List of Figures. The detailed caption appears directly beneath the Figure.

As always, the \verb|\label| command declares the unique key to be used for referring to this Figure.

\begin{figure}[!h]
\centering
\includegraphics[width=0.7\linewidth]{"ch01/Some_Other_Plot"}
\caption[Another figure]{This is a detailed figure caption for another figure, describing what the figure shows. Fun fact: this figure was prepared in RStudio open-source software, just like Figure \ref{fig:plot-label} (yes, you can have a reference to another Figure inside a Figure caption; try doing that in Word!).}
\label{fig:plot2-label}
\end{figure}

\section{A section on tables}\label{tables-sec}
A table can be created pretty easily in \LaTeX{}, as shown in Table \ref{tab:simpletable}. The number of times $\vert$\verb| c |$\vert$ is entered after the \verb|\begin\{tabular}| command declares the number of columns to be used. The \verb|c| stands for ``center'' justification. Other options can be \verb|l| for left, and \verb|r| for right.

As with Figure captioning, there is a short caption (in square brackets; this shows up in List of Tables), and a long caption (in curly brackets; this shows up at the top of the Table). While Figure captions are typically placed below the figure, the convention for Table captions is to place them on top of Tables. This can be achieved by simply moving the \verb|\caption| command as shown in the code for Table \ref{tab:complextable}.

\begin{table}[!h]
\begin{center}
  \begin{tabular}{| c | c | c |}
    \hline
    1 & 2 & 3 \\ \hline
    4 & 5 & 6 \\ \hline
    7 & 8 & 9 \\
    \hline
  \end{tabular}
\caption[A small, simple table]{A table of alphabets. Though this caption is supposed to be detailed, I cannot think of a way to put more text here.}
\label{tab:simpletable}
\end{center}
\end{table}

While Table \ref{tab:simpletable} was fairly simple, a more complicated table with varying column widths can also be created, as shown in Table \ref{tab:complextable}.

\begin{table}[!h]
\caption[A slightly more complex table]{A table of numbers and things. Though this caption is supposed to be detailed, I cannot think of a way to put more text here.}
\label{tab:complextable}
\begin{tabular}{|p{20mm}|p{40mm}|p{30mm}|p{10mm}p{10mm}p{10mm}p{10mm}|}
\hline
\textbf{City}&\textbf{Neighborhood}&\textbf{Time period}&\multicolumn{4}{c|}{\textbf{Results}}\\
&	&	& $\alpha$&$\beta$&$\chi^2$&$\mu$ [m]\\
\hline
Oakland&Full domain&All day&1.83&-0.21&0.89&218\\
 &Downtown&All day&1.24&-0.04&0.03&36\\
 &Urban residential&All day&2.09&-0.23&0.67&234\\
 \hline
Pittsburgh&Full domain&All day&2.18&-0.27&0.52&268\\
 & &Morning&6.42&-0.39&0.42&398\\
 &	 &Midday&0.55&-0.14&0.05&148\\
 & &Afternoon&4.08&-0.42&0.54&419\\
\hline
\end{tabular}
\end{table}

\section{A section on footnotes}\label{sec-footnotes}
I say, ``Footnotes are so easy!''\footnote{And you say, ``How easy are they?!''}

\section{A section on bulleting and numbering}\label{sec-itemizing}
Bulleted lists are very simple to prepare in \LaTeX{}, using the \verb|itemize| environment.
\begin{itemize}
\item An item
\item Another item
\item And so forth
\end{itemize}

The only difference between numbered and bulleted lists is the \verb|enumerate| environment instead of \verb|\itemize|. \LaTeX{} takes care of the numbering during code compilation.
\begin{enumerate}
\item An item
\item Another item
\item And so forth
\end{enumerate}

If you need the numbering reversed, use the \verb|\etaremune| environment\footnote{see the brilliance of it? ``etaremune'' is ``enumerate'' spelled backwards.}.
\begin{etaremune}
\item An item
\item Another item
\item And so forth
\end{etaremune}

\section{A section on chemical reactions}\label{chem-reac-sec}
You can write chemical reactions using the \verb|\chemfig| package:
\begin{center}
\schemestart
\chemfig{H_2O} \arrow{->[$\lambda=185$ nm]}[,1.5] \chemfig{OH} + \chemfig{H}
\schemestop
\end{center}
\begin{center}
\schemestart
\chemfig{O_2} \arrow{->[$\lambda=185$ nm]}[,1.5] 2 O(\textsuperscript{3}P) \arrow{->[\chemfig{O_2}]} 2 O\textsubscript{3} \arrow{->[$\lambda=254$ nm]}[,1.5] 2 O(\textsuperscript{1}D) \arrow{->[\chemfig{H_2O}]} \textit{n}OH
\schemestop
\end{center}

You can also write chemical reactions using the \verb|\chemmacros| package, which follows a slightly different syntax. Unfortunately, neither of them will check your reactions for stoichiometric balancing, so that part's on you.
\begin{center}
\ch{Na2SO4 ->[ H2O ] 2 Na+ + SO4^2-}
%be sure to leave an empty line otherwise the two reactions will be mixed together

\ch{( 2 Na+ ,SO4^2- ) + (Ba^2+ , 2 Cl- ) -> BaSO4 v + 2 NaCl}
\end{center}

You can also include chemical reactions (and organic structures) as a Figure, as shown in Figure \ref{fig:FuncFrag}. \textit{Note:} drawing organic structures can be time-consuming if you have too many of these. If you are writing a hard-core chemical engineering/chemistry document, you might be better off drawing the reactions in other specific software and then import them as figures into your \LaTeX{} document.

\begin{figure}[h]
\centering
\schemestart
\chemfig{OH}
+ \chemfig{R\textsubscript{1}-[:25,0.5]-[:-25,0.5]-[:25,0.5](-[:90,0.5])-[:-25,0.5]-[:25,0.5]-[:-25,0.5]R\textsubscript{2}}
\arrow(reac.mid east--func.mid west){->[functionalization]}[30,2]
\chemname{\chemfig{R\textsubscript{1}-[:25,0.5]-[:-25,0.5](=[:90,0.5]O)-[:25,0.5](-[:90,0.5])-[:-25,0.5](-[:-90,0.5]OH)-[:25,0.5]-[:-25,0.5]R\textsubscript{2}}}{lower volatility}
\arrow(@reac.mid east--frag.mid west){->[fragmentation]}[-30,2]
\chemname{\chemfig{R\textsubscript{1}-[:25,0.5]-[:-25,0.5](=[:90,0.5]O)}
+
\chemfig{-[:25,0.5](=[:90,0.5]O)-[:-25,0.5]-[:25,0.5]-[:-25,0.5]R\textsubscript{2}}}{higher volatility}
\schemestop
\caption[Short caption for chemistry]{Caption describing what is going on in this chemical reaction.}
\label{fig:FuncFrag}
\end{figure}


\newpage
\begin{footnotesize}\singlespacing
\renewcommand{\bibname}{References}
\bibliographystyle{apalike}
\addcontentsline{toc}{section}{References}
\bibliography{ch01/ch01_REFS}
\end{footnotesize}
|). To include your new chapter, just enter another line: \verb|\include{ch03/ch03}|.

Finally, go to the \verb|ThesisTemplate| folder, right-click the \verb|MAC_MakeThesis.sh| file (or \verb|WIN_MakeThesis.bat|), and open with a text editor like TextEdit in Mac, or Notepad in Windows. Find the chapter-specific lines of code (e.g., \verb|bibtex ch02/ch02.tex| and \verb|bibtex ch02/ch02.aux|), copy-paste them in a new line, and edit them accordingly for your new chapter. Hit save, but don't close the file yet...

\textit{Optional, but recommended:} In the execution file, there is ``chapter-specific cleanup'' section. You can copy the chapter-specific lines of code, paste them in a new line, and edit them accordingly for your new chapter. This will make sure that all auxilary files are deleted after the code has finished compiling. Hit save, and now you may close the execution file.

And now, if you double-click the execution file you just edited, it should compile your entire thesis (give it a few seconds), and the final thesis should open automatically in your default PDF reader.
\end{document}